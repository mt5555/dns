\documentclass[12pt]{article}
\usepackage{amssymb,latexsym,amsmath}

\newif\ifpdf
\ifx\pdfoutput\undefined
\pdffalse % we are not running PDFLaTeX                                         
\else
\pdfoutput=1 % we are running PDFLaTeX                                          
\pdftrue
\fi
\ifpdf
\usepackage[pdftex]{graphicx}
\else
\usepackage{graphicx}
\fi
\ifpdf
\DeclareGraphicsExtensions{.pdf, .jpg, .tif}
\else
\DeclareGraphicsExtensions{.eps, .jpg}
\fi



\textwidth = 6.5 in
\textheight = 9 in
\oddsidemargin = 0.0 in
\evensidemargin = 0.0 in
\topmargin = 0.0 in
\headheight = 0.0 in
\headsep = 0.0 in
\parskip = 0.2 in
\parindent = 0.0 in


%%%%%%%%%%%%%%%%%%%%%%%%%%%%%%%%%%%%%%%%%%%%%%%%%%%%%%%%%%%%%%%%%%%%%%%%


\newcommand{\khat}{\hat{\mathbf k}}
\newcommand{\uv}{\mathbf u}
\newcommand{\up}{\mathbf u'}
\newcommand{\w}{\mathbf w}
\newcommand{\grad}{\nabla}
\newcommand{\curlp}{\gradp \times}
\newcommand{\curl}{\grad \times}
\newcommand{\gradp}{\nabla'}

\DeclareMathOperator{\Span}{span}


\title{Coordinate scaling in the LANL DNS code}
\author{Mark Taylor}

\begin{document}
%\maketitle

\section{Scaling of the Domain}
\begin{figure}
\begin{center}
\includegraphics[angle=-90,width=4in]{box}
\caption{ }
\label{F:box}
\end{center}
\end{figure}



We would like to solve the NS equations in the box pictured in
Fig.~\ref{F:box}.  To write the equations in this domain, we use
primes:
\[
\frac{ \partial  \up }{\partial t}  + (\curlp \up + f \khat) \times \up + 
\gradp \pi' + N \theta' \khat = \nu \Delta' \up
\]
\[
\gradp \cdot \up = 0
\]
\[
\frac{ \partial  \theta' }{\partial t}  + \up \cdot \grad' \theta' - 
N \up \cdot \khat = \kappa \Delta' \theta'
\]
with
\[
\gradp = 
\begin{pmatrix} \dfrac{\partial}{\partial x'} \\[5mm]
                \dfrac{\partial}{\partial y'} \\[5mm]
                \dfrac{\partial}{\partial z'} 
\end{pmatrix}
 \qquad
\Delta'  = \left(\frac{\partial}{\partial x'}\right)^2 + 
           \left(\frac{\partial^2}{\partial y'}\right)^2 +
           \left(\frac{\partial^2}{\partial z'}\right)^2 
\]
The Schmidt number is $\nu / \kappa$.  

We now introduce a change of variables, into the {\em non-primed} coordiante
system.  In the non-primed coordinate system, the domain because a 
unit cube:
\[
x = x', \qquad y=y', \qquad  z = \frac{z'}{h},
\]
\[
u_1 = u_1', \qquad u_2 = u_2', \qquad  h u_3 = u_3'
\]
and
\[
h \theta = \theta'
\]
And we also have:
\[
\frac{\partial}{\partial z'} = \frac{1}{h} \frac{\partial}{\partial z}
\]
We first write $\gradp \cdot \up$ in the non-primed coordinate system:
\[
\frac{\partial u_1' }{\partial x'} = \frac{\partial u_1 }{\partial x}
\qquad
\frac{\partial u_2' }{\partial y'} = \frac{\partial u_2 }{\partial y}
\qquad
\frac{\partial u_3' }{\partial z'} = \frac{1}{h} \frac{\partial h u_3 }{\partial z}
= \frac{\partial u_3 }{\partial z}
\]
So we have that
\[
\gradp \cdot \up = \grad \cdot \uv
\]
And thus the divergence free condition is the 
same in both coordinate systems.  The term $\gradp \pi'$ is the
unique gradient which maintains $\gradp \cdot \up = 0$.  This
term must be expressible as a gradient in physical coordinates,
so we can drop the prime, but will must still write the term as
$\gradp \pi$, where now it is the unique (physical coordinate) 
gradient which maintains  $\grad \cdot \uv = 0$.
We now start substituting the non-primed variables with the
primted variables:
\[
\begin{pmatrix} u_1 \\
                u_2 \\
                h u_3 \\
\end{pmatrix}_t
  + (\curlp \up + f \khat) \times 
\begin{pmatrix} u_1 \\
                u_2 \\
                h u_3 \\
\end{pmatrix}
+ \gradp \pi +
\begin{pmatrix} 0 \\
                0 \\
                h N \theta \\
\end{pmatrix}
= \nu 
\begin{pmatrix} \dfrac{\partial^2 u_1}{\partial x^2} + 
                \dfrac{\partial^2 u_1}{\partial y^2} + 
                \dfrac{1}{h^2}\dfrac{\partial^2 u_1}{\partial z^2}   \\[5mm]
                \dfrac{\partial^2 u_2}{\partial x^2} + 
                \dfrac{\partial^2 u_2}{\partial y^2} + 
                \dfrac{1}{h^2}\dfrac{\partial^2 u_2}{\partial z^2}   \\[5mm]
                h \left( 
                \dfrac{\partial^2 u_3}{\partial x^2} + 
                \dfrac{\partial^2 u_3}{\partial y^2} + 
                \dfrac{1}{h^2}\dfrac{\partial^2 u_3}{\partial z^2} \right)
\end{pmatrix}
\]
\[
h \frac{ \partial  \theta }{\partial t}  + h \uv \cdot \grad \theta - 
N h u_3  = h \kappa \left(
                \dfrac{\partial^2 \theta}{\partial x^2} + 
                \dfrac{\partial^2 \theta}{\partial y^2} + 
                \dfrac{1}{h^2}\dfrac{\partial^2 \theta}{\partial z^2}
\right)
\]

Defining $\omega'$ by  
\begin{equation}
\label{E:vor}
\omega = \curlp \up = 
\begin{pmatrix}
h \dfrac{\partial u_3}{\partial y} - \dfrac{1}{h} \dfrac{\partial u_2}{\partial z}  \\[5mm]
\dfrac{1}{h} \dfrac{\partial u_1}{\partial z} -  h \dfrac{\partial u_3}{\partial x}  \\[5mm]
\dfrac{\partial u_2}{\partial x} - \dfrac{\partial u_1}{\partial y}
\end{pmatrix}  
\end{equation}
the vorticity term works out to 
\[
(\curlp \up + f\khat) \times 
\begin{pmatrix} u_1 \\
                u_2 \\
                h u_3 \\
\end{pmatrix} = 
\begin{pmatrix}  h  u_3 \omega_2' -  u_2 \omega_3'   \\
                   u_1 \omega_3' -   h u_3 \omega_1'   \\
                u_2 \omega_1' - u_1 \omega_2'   \\
\end{pmatrix}  
+ 
\begin{pmatrix} -f u_2 \\
                f u_1 \\
                0  \\
\end{pmatrix}  
\]

Thus the final equations, after scaling the $z$ component by $h$
and substituting in for $\omega$:
\[
\begin{pmatrix} u_1 \\
                u_2 \\
                u_3 \\
\end{pmatrix}_t
  + 
\begin{pmatrix}  h  u_3 \omega_2' -  u_2 \omega_3'   \\
                   u_1 \omega_3' -   h u_3 \omega_1'   \\
              \dfrac{1}{h} \left(  u_2 \omega_1' - u_1 \omega_2' \right)  \\
\end{pmatrix}  
+ 
\begin{pmatrix} -f u_2 \\
                f u_1 \\
                N \theta  \\
\end{pmatrix}  
+ \gradp \pi = \nu 
\begin{pmatrix} \dfrac{\partial^2 u_1}{\partial x^2} + 
                \dfrac{\partial^2 u_1}{\partial y^2} + 
                \dfrac{1}{h^2}\dfrac{\partial^2 u_1}{\partial z^2}   \\[5mm]
                \dfrac{\partial^2 u_2}{\partial x^2} + 
                \dfrac{\partial^2 u_2}{\partial y^2} + 
                \dfrac{1}{h^2}\dfrac{\partial^2 u_2}{\partial z^2}   \\[5mm]
                \dfrac{\partial^2 u_3}{\partial x^2} + 
                \dfrac{\partial^2 u_3}{\partial y^2} + 
                \dfrac{1}{h^2}\dfrac{\partial^2 u_3}{\partial z^2}
\end{pmatrix}
\]
\[
\frac{ \partial  \theta }{\partial t}  + \uv \cdot \grad \theta - 
N u_3  = \kappa \left(
                \dfrac{\partial^2 \theta}{\partial x^2} + 
                \dfrac{\partial^2 \theta}{\partial y^2} + 
                \dfrac{1}{h^2}\dfrac{\partial^2 \theta}{\partial z^2}
\right)
\]



Thus the only changes needed for our parallel DNS code are:
\begin{enumerate}
\item initial velocity given in the $\up$ coordinate system 
needs to be scaled to the non-primed coordinate system $\uv$.
\item Our output files will contain velocity will be in the non-primed 
$\uv$ coordinate system.  Need to scaled appropriately  to be analysed
by other codes.  We have added a new output file header type that
contains h.  
\item modify computation of vorticity to match Eq.~\ref{E:vor}.
\item modify Laplacian (scale $z$ derivative terms by $1/h^2$).
\item modify vorticity term $\w \times \uv$ as shown above.
\end{enumerate}

\section{Hyper Viscosity}
Consider our scaled hyper viscosity term
\[
\frac{ \partial  \up }{\partial t} = 
\sqrt{E(k_\textrm{max})} \,  k_\textrm{max}^{-2\eta + 1.5}  \Delta^\eta \up
\]
We first verify that the units are correct.  The units of
energy are $m^2/s^2$, so the units of the energy in a single
spherical shell $E(k)$ is $m^3/s^2$.  Thus we have
\[
\frac{m}{s^2} = \frac{m^{1.5}}{s} \,\frac{1}{m^{-2\eta+1.5}}  
                          \,\frac{1}{m^{2\eta}}
                           \, \frac{m}{s}
\]
In the frequency domain, 
\[
\frac{ \partial  u_3 }{\partial t} = 
\sqrt{E(k_\textrm{max})} \,  k_\textrm{max}^{-2\eta + 1.5}  
( (2\pi l)^2 + (2\pi m)^2 + (2\pi n/h)^2)^\eta u_3
\]

\section{Resolution Condition}

The assumptions made in the hydrostatic equations implicitly assume
(acording to Leslie) that 
\[
 \max{ k_h } << \min{k_z}
\]
which works out to
\[
   \frac{\sqrt{2}}{3} N_x <<  2 \pi /h.
\]



\section{Computing Scalars}
Integrals per unit volume are approximated by the sum
\[
<f,g> = \frac{1}{h} \iiint  f g  \, dx' \, dy' \, dz' =   \iiint  f g \, dx \, dy \, dz
=\frac{1}{N_x N_y N_z} \sum  f  g
\]
We also have that 
\[
\frac{1}{N_x N_y N_z} \sum  f  g = \sum \hat{f} \hat{g}
\]
so integrals can be computed in Fourier or grid space. 

All other scalars (kinetic energy, helicity, various dissipation
terms) are computed in physical coordinates.  So for example
\[
k = <\up,\up> = <u_1,u_1> + <u_2,u_2> + h^2 <u_3,u_3>
\]
and the helicity 
\[
  h  = w_1' u_1  + w_2' u_2 + h w_3' u_3
\]
As in example, using integer wave numbers $(l,m,n)$ (defined in the
next section), we have
\[
< \omega' , \Delta \up> = \sum (2 \pi)^2 (l^2 + m^2 + (n/h)^2) 
\begin{pmatrix} \hat{u}_1 \\
                \hat{u}_2 \\
                h \hat{u}_3  \\
\end{pmatrix}  
 \cdot \omega'
\]





\section{Computing 3D Energy Spectra (in a spherical shell)}
If we choose to discretize our domain so that the 
maximum wave number (in the primed coordinate system) is
the same in all diretions, then
\[
\Delta x' = \Delta y' = \Delta z'
\]
Or
\[
N_x  = N_y = N_z/h
\]
where $N_x$ is the number of grid points in the $x$ direction.
The highest wave number is given by:
\[
k_x = \pi N_x, \qquad k_y = \pi N_y, \qquad k_z = \pi N_z/h 
\]
(Note that $\pi$ appears in the wave numbers since our domain is
of length 1 in the $x$ and $y$ diretions.)  The wave number spacing
is given by
\[
\Delta k_x = 2 \pi, \qquad  \Delta k_y = 2 \pi, \qquad  \Delta k_z = 2 \pi / h 
\]
In our DNS code, we index our arrays of Fourier coefficients
with integers $(l,m,n)$, with $l=0 \dots N_x/2$, 
$m=0 \dots N_y/2$ and 
$n=0 \dots N_z/2$.  The wave number spacing is given by
given by
\[
k_x = l \Delta k_x,, \qquad  k_y = m \Delta k_y, \qquad  k_z = n  \Delta k_z
\]
We index the arrays containing power spectra
such as $E(k)$ with integer wave number $k$, represents a shell or anulus
of some chosen thickness centered around the wave number associated
with $k$.

The total energy is
\[
E  = \sum_{l,m,n}  |  {\hat u_1}(l,m,n) |^2 +
 |  {\hat u_2}(l,m,n) |^2 + 
 |  h {\hat u_3}(l,m,n) |^2
\]

The standard energy spectrum is now, for integers $k$:
\[
E(k)  = \sum_{l,m,n}  |  {\hat u_1}(l,m,n) |^2 +
 |  {\hat u_2}(l,m,n) |^2 + 
 |  h {\hat u_3}(l,m,n) |^2
\]
where the sum is taken over all integers $(l,m,n)$ such that
\[
(\Delta k_z)^2 (k-1/2)^2 \le (\Delta k_x l)^2 + (\Delta k_y m)^2 + (\Delta k_z n)^2 <  (\Delta k_z)^2 (k+1/2)^2 
\]
where we have chosen to use a shell thickness of $\Delta k_z$.
 Scaling out the $2\pi$, we get
\[
\frac{(k-1/2)^2}{h^2} \le l^2 + m^2 + \frac{n^2}{h^2} < \frac{(k+1/2)^2}{h^2}
\]  




\section{Computing 2D Energy Spectra (in an annulus)}
We also compute a 2D spectra $E(k_h,k_z)$ by summing squares of Fourier
coefficients over annular regions in 
$x$ and $y$ of radius $k_h$.  
where 
\[
k_h = \sqrt{(k_x^2 + k_y^2)} = \sqrt{(2\pi l)^2 + (2\pi m)^2}. 
\]
From this quantity, we can get $E(k_h,0)$, which 
represents the energy spectrum of the 2D field
obtained by averaging the original 3D field  in the direction of rotation,
and then summing the energy within spherical shells (an annulus for 2D fields).
We are also interested in 
\begin{equation}
E(k_h) = \sum_{k_z=0}^{\pi N_z/h} E(k_h,k_z) 
\label{E:SPECb}
\end{equation}
which represents 
the energy spectrum of the full 3D field, but summed over
the annular region between concentric cylinders
(whose axis are in the $z$ direction).  As with the spherical
shell spectrum, Eq.~\ref{E:SPECb} has the property that
it partitions the energy into a set of discrete wave numbers, so
that the total energy 
\[
E = \sum_{k_h} E(k_h)
\]


The 2D shell (annulus) thickness is now
determined by the horizontal wavenumber spacings. In our code,
we use an integer index $k$ defined by $k_h = k \Delta k_x$ and
define
\begin{eqnarray*}
E(k,n) = \sum_{l,m}  |  {\hat u_1}(l,m,n) |^2 +
 |  {\hat u_2}(l,m,n) |^2 + 
 |  h {\hat u_3}(l,m,n) |^2
\end{eqnarray*}
where the sum is taken over fixed $n$ and all integers $(l,m)$ such that 
\[
(\Delta k_x)^2 (k-1/2)^2 \le (\Delta k_x l)^2 + (\Delta k_y m)^2  <  (\Delta k_x)^2 (k+1/2)^2 
\]
and with $k_x=k \Delta k_x$ for integers $k$.  
we have chosen to use a shell thickness of $\Delta k_x = \Delta
k_y = 2\pi$. Scaling out the $2\pi$, we get
\[
(k-1/2)^2 \le l^2 + m^2 < (k+1/2)^2.
\]  



\end{document}