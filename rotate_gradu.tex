\documentclass[12pt]{article}
\usepackage{amssymb,latexsym,amsmath,bm}

\newif\ifpdf
\ifx\pdfoutput\undefined
\pdffalse % we are not running PDFLaTeX
\else
\pdfoutput=1 % we are running PDFLaTeX
\pdftrue
\fi
\ifpdf
\usepackage[pdftex]{graphicx}
\else
\usepackage{graphicx}
\fi
\ifpdf
\DeclareGraphicsExtensions{.pdf, .jpg, .tif}
\else
\DeclareGraphicsExtensions{.eps, .jpg}
\fi



\textwidth = 6.5 in
\textheight = 9 in
\oddsidemargin = 0.0 in
\evensidemargin = 0.0 in
\topmargin = 0.0 in
\headheight = 0.0 in
\headsep = 0.0 in
\parskip = 0.2 in
\parindent = 0.0 in


%%%%%%%%%%%%%%%%%%%%%%%%%%%%%%%%%%%%%%%%%%%%%%%%%%%%%%%%%%%%%%%%%%%%%%%%


\newcommand{\khat}{\hat{\mathbf k}}
\newcommand{\uv}{\mathbf u}
\newcommand{\up}{\mathbf u'}
\newcommand{\w}{\mathbf w}
\newcommand{\grad}{\nabla}
\newcommand{\curlp}{\gradp \times}
\newcommand{\curl}{\grad \times}
\newcommand{\gradp}{\nabla'}

\DeclareMathOperator{\Span}{span}

\title{Rotation matrix for transforming the strain tensor}

\begin{document}
%\maketitle
\section*{Shear tensor}
There are 9 components of the shear (strain rate) $S_{ij} = \grad {\bm u} =
\partial_j u_i$. We can compute all these components in the
grid-coordinate system ($\bm{{\hat x}}_1,\bm{{\hat x}}_2, \bm{{\hat
x}}_3$). We would like to be able to say what these components are in
a coordinate system whose $x_1$-axis is along $\bm{{\hat r}}$, one of
the 73 directions we compute in our angle averaging procedure.

\section*{New coordinate system}
Let $\bm{{\hat r}}$, $\bm{{\hat t}}_1$
and $\bm{{\hat t}}_2$ (which are determined by the existing code in the grid-coordinate system) form our new coordinate system.
The axis of the new system $\bm{{\hat x}}_1',\bm{{\hat x}}_2',
\bm{{\hat x}}_3'$ are obviously 
given in the transformed coordinate system as $\bm{{\hat x}}_1' =
\left(\begin{array}{c} 1 \\ 0 \\0
\end{array}\right)$, $\bm{{\hat x}}_2' = \left(\begin{array}{c} 0 \\ 1
\\0 \end{array}\right)$ and $\bm{{\hat x}}_3' = \left(\begin{array}{c}
0 \\ 0 \\1 \end{array}\right)$.

The transformation between the two coordinate systems is governed by
the rotation matrix ${\bf A}$ which can be written in terms of the
known components of the new coordinate system. That is, we have 

$$\bm{{\hat x}}_1' = \left(\begin{array}{c} 1 \\ 0 \\0\end{array}\right) = {\bf A}\left(\begin{array}{c} \bm{{\hat r}}\cdot \bm{{\hat x}}_1 \\ \bm{{\hat r}}\cdot \bm{{\hat x}}_2 \\\bm{{\hat r}}\cdot \bm{{\hat x}}_3\end{array}\right);~~
\bm{{\hat x}}_2' = \left(\begin{array}{c} 0 \\ 1 \\0\end{array}\right) = {\bf A}\left(\begin{array}{c}\bm{{\hat t}}_1\cdot \bm{{\hat x}}_1 \\ \bm{{\hat t}}_1 \cdot\bm{{\hat x}}_2 \\\bm{{\hat t}}_1 \cdot \bm{{\hat x}}_3 \end{array}\right);~~
\bm{{\hat x}}_3' = \left(\begin{array}{c} 0 \\ 0 \\1 \end{array}\right)= {\bf A}\left(\begin{array}{c}\bm{{\hat t}}_2\cdot \bm{{\hat x}}_1 \\ \bm{{\hat t}}_2 \cdot\bm{{\hat x}}_2 \\\bm{{\hat t}}_2 \cdot \bm{{\hat x}}_3 \end{array}\right)$$


This gives the rotation matrix 

$${\bf A} = \left(\begin{array}{ccc}  
\bm{{\hat r}}\cdot \bm{{\hat x}}_1 & \bm{{\hat r}}\cdot \bm{{\hat x}}_2 &\bm{{\hat r}}\cdot \bm{{\hat x}}_3 \\
\bm{{\hat t}}_1\cdot \bm{{\hat x}}_1 & \bm{{\hat t}}_1 \cdot\bm{{\hat x}}_2 & \bm{{\hat t}}_1 \cdot \bm{{\hat x}}_3 \\
\bm{{\hat t}}_2\cdot \bm{{\hat x}}_1 & \bm{{\hat t}}_2 \cdot\bm{{\hat x}}_2 & \bm{{\hat t}}_2 \cdot \bm{{\hat x}}_3 
\end{array} 
\right)$$

(check {${\bf A A}^{T} = {\bf I}$)


So, the strain tensor in the new coordinate system is  given by 
${\bf S}' = {\bf A S A}^{T}$, or, component-wise
$S_{ij}' = A_{ik} S_{kl} A_{jl}$.

We cycle through the 73 different $\bm{{\hat r}}$ in this way to find the one 
which give the largest off-diagonal
contribution to $\grad \bm{u}$ (either $S_{12}$ or $S_{13}$, and
compute the mixed structure function $\langle \Delta u_1 (r) \Delta u_2 (r)\rangle$ and 
$\langle \Delta u_1 (r) \Delta u_3 (r)\rangle$ for that direction of $\bm{{\hat r}}$.

Redundancies....

\end{document}