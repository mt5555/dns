\documentclass[pre,preprint,showpacs]{revtex4}
% \documentclass{article}
\usepackage{graphicx}
\begin{document}
\title{forcing}
\author{Mark A. Taylor}


\subsection{Deterministic forcing \label{forcing_det}}
We make use of a determinisitc low wave number forcing, modeled after
the schemes described in \cite{Kerr94,SVBSCC96,OvePop98}, where the
energy in a few low wave numbers is relaxed back to a target spectrum.
Here we instead relax the flow at each time step towards a configuratin
with maximum helicity, as in \cite{Kurien04}, which is based on
that first used in \cite{PolSht89}

We force only those modes in the first wave number shell, 
$\tilde {\bf u}_{\bf k}$ for $0.5 \le |{\bf k}| < 1.5$, 
where $\tilde {\bf u}_{\bf k}$ is the $k$th Fourier coefficient of
${\bf u}$.
We define the energy in this shell in 
the usual way:
\begin{equation}
E(1) = \sum_{0.5 \le |{\bf k}| < 1.5} \frac12 | \tilde {\bf u}_{\bf k} | ^2.
\end{equation}
Let ${\bf u'}$ denote the truncated ${\bf u}$,
\begin{equation}
 \tilde {\bf u'}_{\bf k} = \tilde {\bf u}_{\bf k}, \quad 0.5 \le |{\bf k}| <1.5,
\qquad {\bf u'}_{\bf k}=0 \quad \text{otherwise}.
\end{equation}
The flow will be relaxed back to a field
${\bf v}$, chosen so that 
$| \tilde {\bf v}_{\bf k}| = |\tilde {\bf u}'_{\bf k}| $
and the phases of $\tilde {\bf v}_{\bf k}$ are chosen so that $\bf v$ has
maximum helicity, using the procedure from \cite{PolSht89}. 
Note that $< {\bf u}' \cdot {\bf u}' > = < {\bf v} \cdot {\bf v} >  = 2 E(1)  $. 


The simplest form of this forcing term would be
\begin{equation}
{\bf f} = \tau_1 ( {\bf v} -  {\bf u'} )
\end{equation}
and $\tau_1$ is chosen at each timestep in order to obtain a constant
energy injection rate.  This forcing function has 
the drawback that as ${\bf u}$ approaches maximal helicity, 
${\bf v} -  {\bf u'}$ will approach zero, forcing 
$\tau_1$ to approach infinity, creating a very short
relaxation time scale which requires a very small time step
to resolve.  We overcome this drawback
by introducing an addition term,
\begin{equation}
{\bf f} = \tau_1 ( {\bf v} -  {\bf u'} )  + \tau_2 {\bf u'}.
\end{equation}
With this form, we can choose $\tau_1 = 1/\Delta t$, 
the largest numerically stable
value, and then take $\tau_2$ to insure a constant energy injection rate
$<   {\bf u} \cdot {\bf f}  > $.
Note that
\begin{equation}
<  {\bf u} \cdot {\bf f} > = \tau_1 < {\bf u'} \cdot {\bf v} >
-  2 \tau_1 E(1)   + 2 \tau_2 E(1)
\end{equation}
so that $\tau_2$ is given by
\begin{equation}
2 \tau_2 E(1)  = < {\bf u} \cdot {\bf f} >  +  
 \tau_1  \left( 2 E(1)    -    <  {\bf u'} \cdot {\bf v} >  \right ).
\end{equation}
Using the inequality
\begin{equation}
<  {\bf u'} \cdot {\bf v} > \le \left( <  {\bf u'} \cdot  {\bf u'} > 
                                <   {\bf v} \cdot  {\bf v} > \right)^{1/2} = 2 E(1),
\end{equation}
we can see that $\tau_2>0$ and thus the forcing will always be the sum
of a term $\tau_1 ( {\bf v} -  {\bf u'} )$, 
which relaxes the solution towards a maximal helicity configuration, 
and a term $ \tau_2 {\bf u'}$, 
which increases the energy while preserving the phases.  This later
term is most active only when the forced modes of ${\bf u}$ have 
large helicity.  

\begin{thebibliography}{99}

\bibitem{SVBSCC96} K.R. Sreenivasan, S.I. Vainshtein, R. Bhiladvala, 
I. SanGil, S. Chen and N. Cao, Phys. Rev. Lett., {\bf 77} pp. 1488--1491 
(1996). 

\bibitem{OvePop98}  M. R. Overholt and S. B. Pope,
Comp. Fluids, {\bf 27} pp. 11--28 (1998).   

\bibitem{Kerr94}  N. P. Sullivan, S. Mahalingam, R. M. Kerr,
Phys. Fluids, {\bf 6} pp. 1612--1614 (1994).  

\bibitem{Kurien04}  S. Kurien, M. A. Taylor, T. Matsumoto,
{\it Cascade time scales for energy and helicity in homogeneous
isotropic turbulence,}  Phys. Rev. E, {\bf 69} (2004) 

\bibitem{PolSht89} W. Polifke and L. Shtilman, 
{\it The dynamics of helical decaying turbulence},
Phys. Fluids A, {\bf  1} pp. 2025­2033 (1989) 


\end{thebibliography}

\end{document}
